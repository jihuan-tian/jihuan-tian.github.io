% Created 2023-08-08 Tue 21:54
% Intended LaTeX compiler: pdflatex
\documentclass[11pt]{article}
\input{page-size}
\input{generic}
\input{math}
\input{font-selection}
\author{Jihuan Tian}
\date{\today}
\title{Construct topological data structures from a mesh}
\hypersetup{
 pdfauthor={Jihuan Tian},
 pdftitle={Construct topological data structures from a mesh},
 pdfkeywords={},
 pdfsubject={},
 pdfcreator={Emacs 27.1 (Org mode 9.4)}, 
 pdflang={English}}
\begin{document}

\maketitle
\setcounter{tocdepth}{5}
\tableofcontents

When we construct topological data structures from a mesh, such as the signed incidence matrix associating \(k\)-simplices and \(k+1\)-simplices therein, don't be keen to directly achieve this target in one round. Sometimes, we need several intermediate steps and data structures as a scaffold for the construction. For example, we can firstly create a collections of dynamic vectors storing the incident edges of each cell in the mesh, then transform these data into the final compressed sparse row (CSR) matrix format. We can see that to realize the final algorithm with the minimal storage and time complexity, it is worthwhile to implement complex mechanisms with affordable overheads. There is actually no free lunch!

incidence [OED] The situation of one locus with respect to another when they have a common point or points, but do not completely coincide; e.g. of a point to a line on which it lies, of a point or a line to a plane in which it lies, or of two intersecting lines to each other. [From the German of Schubert, Kalkul der Abzähl. Geom. (1879) 25.]
\end{document}